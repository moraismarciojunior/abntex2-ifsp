\PassOptionsToPackage{inline,shortlabels}{enumitem}
\documentclass[12pt, oneside, a4paper, english, brazil]{abntex2}

\usepackage[logohorizontal, CAR]{abntex2-ifsp} 
% ---
		
% ---
\usepackage{amsmath}
\usepackage{amsfonts}
\usepackage{amssymb}
\usepackage{amsthm}

\usepackage{xpatch}

\usepackage{enumerate} 
\usepackage{graphicx}
\usepackage{tikz}



\providecommand{\abs}[1]{\left\vert #1 \right\vert}
\providecommand{\p}[1]{\left( #1 \right)}
\providecommand{\chaves}[1]{\left\{ #1 \right\}}
\providecommand{\colchetes}[1]{\left[ #1 \right]}
\providecommand{\Endo}[1]{\text{End}\left( #1 \right)}
\providecommand{\Endor}[1]{\text{End}_{\mathbb{R}}\left( #1 \right)}


\providecommand{\R}{\mathbb{R}}
\providecommand{\Rdois}{\mathbb{R}^2}
\providecommand{\Rtres}{\mathbb{R}^3}
\newcommand{\C}{\mathbb{C}}
%\newcommand{\bs}{\backslash}
\newcommand{\vep}{\varepsilon}
\providecommand{\rad}{\text{ rad}}

\providecommand{\ac}{Álgebras de Clifford}
\providecommand{\Cl}{\mathcal{C}\ell}
\providecommand{\Cldois}{\mathcal{C}\ell_2}
\providecommand{\Cldoispar}{\mathcal{C}\ell_2^{+}}
\providecommand{\Cldoisimpar}{\mathcal{C}\ell_2^{-}}

\providecommand{\pref}[1]{(\ref{#1})}
\providecommand{\bref}[1]{[\ref{#1}]}

\renewcommand{\vec}{\overrightarrow}
\renewcommand{\qedsymbol}{$\blacksquare$}

% ---

% ---
% Definições, Teoremas, Corolários ...
% ---
\providecommand{\definitionref}[1]{[Definição \ref{#1}]}
\providecommand{\propositionref}[1]{[Proposição \ref{#1}]}
\providecommand{\propertyref}[1]{[Propriedade \ref{#1}]}
\providecommand{\observationref}[1]{[Observação \ref{#1}]}
\providecommand{\corollaryref}[1]{[Corolário \ref{#1}]}
\providecommand{\lemmaref}[1]{[Lema \ref{#1}]}
\providecommand{\theoremref}[1]{[Teorema \ref{#1}]}
\providecommand{\conjectureref}[1]{[Conjectura \ref{#1}]}
\providecommand{\exempleref}[1]{[Exemplo \ref{#1}]}
\providecommand{\notationref}[1]{[Notação \ref{#1}]}

\xpatchcmd{\proof}{\hskip\labelsep}{\hskip6.5\labelsep}{}{}

\newtheoremstyle{normal}% name
{}% Space above1
{}% Space below1
{}% Body font
{\parindent}% Indent amount2
{\bfseries}% Theorem head font
{.}% Punctuation after theorem head
{ }% Space after theorem head3
{}% Theorem head spec (can be left empty, meaning ‘normal’)

\newtheoremstyle{observacao} 
{}% Space above1
{}% Space below1
{}% Body font
{\parindent}% Indent amount2
{\bfseries}% Theorem head font
{:}% Punctuation after theorem head
{ }% Space after theorem head3
{}% Theorem head spec (can be left empty, meaning ‘normal’)


\newcounter{geral}

\theoremstyle{normal}
\newtheorem{definition}[geral]{Definição}
\newtheorem{proposition}[geral]{Proposição}
\newtheorem{property}[geral]{Propriedade}
\newtheorem{corollary}[geral]{Corolário}
\newtheorem{lemma}[geral]{Lema}
\newtheorem{theorem}[geral]{Teorema}
\newtheorem{conjecture}[geral]{Conjectura}
\newtheorem{notation}[geral]{Notação}
\newtheorem{vocabulary}[geral]{Vocabulário}

\theoremstyle{observacao}
\newtheorem*{obs}{Observação}


% ---
% Informações de dados para CAPA e FOLHA DE ROSTO
% ---
\titulo{As Álgebras de Clifford}
\autor{Marcio Oliveira de Morais Junior}
%\local{Brasil}
\data{2016}
\datadefesa{\today}
\orientador{Prof. Me. Márcio André Traesel}
%\coorientador{Um coorientador}
\membrobancaA{Prof. Me. }
\membrobancaB{Prof. Me. }


\tipotrabalho{Trabalho de Conclusão de Curso}
\preambulo{Monografia submetida ao Instituto Federal de Educação, Ciência e Tecnologia de São Paulo -- Câmpus Caraguatatuba como parte dos requisitos para obtenção do grau de Licenciado em Matemática.}
% ---


% ---
% Configurações de aparência do PDF final

% alterando o aspecto da cor azul
\definecolor{blue}{RGB}{41,5,195}

% informações do PDF
\makeatletter
\hypersetup{
%pagebackref=true,
pdftitle={\@title}, 
pdfauthor={\@author},
pdfsubject={\imprimirpreambulo},
pdfcreator={LaTeX with abnTeX2},
pdfkeywords={Álgebras de Clifford}{Álgebras Geométricas}{Matemática Pura}{Matemática Aplicada}, 
colorlinks=false,       		% false: boxed links; true: colored links
linkcolor=blue,          	% color of internal links
citecolor=blue,        		% color of links to bibliography
filecolor=magenta,      	% color of file links
urlcolor=blue,
bookmarksdepth=4
}
\makeatother
% --- 

% ---
% compila o indice
% ---
\makeindex
% ---

% ----------------------------------------------------------
% Início do documento
% ----------------------------------------------------------
\begin{document}

% ----------------------------------------------------------
% ELEMENTOS PRÉ-TEXTUAIS
% ----------------------------------------------------------
\pretextual

% ----------------------------------------------------------
% Capa
% ----------------------------------------------------------
\imprimircapa


% ----------------------------------------------------------
% Folha de rosto
% (o * indica que haverá a ficha bibliográfica)
% ----------------------------------------------------------
\imprimirfolhaderosto*


% ----------------------------------------------------------
% Inserir a ficha bibliografica
% ----------------------------------------------------------

% Isto é um exemplo de Ficha Catalográfica, ou ``Dados internacionais de
% catalogação-na-publicação''. Você pode utilizar este modelo como referência. 
% Porém, provavelmente a biblioteca da seu câmpus lhe fornecerá um PDF
% com a ficha catalográfica definitiva após a defesa do trabalho. Quando estiver
% com o documento, salve-o como PDF no diretório do seu projeto e substitua todo
% o conteúdo de implementação deste arquivo pelo comando abaixo:
%
% \begin{fichacatalografica}
%     \includepdf{fichacatalografica.pdf}
% \end{fichacatalografica}
\begin{fichacatalografica}
\vspace*{\fill}					% Posição vertical
\hrule							% Linha horizontal
\begin{center}					% Minipage Centralizado
\begin{minipage}[c]{12.5cm}		% Largura

\imprimirautor

\hspace{0.5cm} \imprimirtitulo  / \imprimirautor. --
\imprimirlocal, \imprimirdata-

\hspace{0.5cm} \pageref{LastPage} p. : il. (algumas color.) ; 30 cm.\\

\hspace{0.5cm} \imprimirorientadorRotulo~\imprimirorientador\\

\hspace{0.5cm}
\parbox[t]{\textwidth}{\imprimirtipotrabalho~--~\imprimirinstituicao,
\imprimirdata.}\\

\hspace{0.5cm}
1. Palavra-chave1.
2. Palavra-chave2.
I. Orientador.
II. Universidade xxx.
III. Faculdade de xxx.
IV. Título\\ 			
	
\hspace{8.75cm} CDU 02:141:005.7\\
	
\end{minipage}
\end{center}
\hrule
\end{fichacatalografica}


% ----------------------------------------------------------
% Inserir errata
% ----------------------------------------------------------
%\begin{errata}
%Elemento opcional da \citeonline[4.2.1.2]{NBR14724:2011}. Exemplo:
%
%\vspace{\onelineskip}
%
%FERRIGNO, C. R. A. \textbf{Tratamento de neoplasias ósseas apendiculares com
%reimplantação de enxerto ósseo autólogo autoclavado associado ao plasma
%rico em plaquetas}: estudo crítico na cirurgia de preservação de membro em
%cães. 2011. 128 f. Tese (Livre-Docência) - Faculdade de Medicina Veterinária e
%Zootecnia, Universidade de São Paulo, São Paulo, 2011.
%
%\begin{table}[htb]
%\center
%\footnotesize
%\begin{tabular}{p{1.4cm}|p{1cm}|p{3cm}|p{3cm}}
%\hline
%\textbf{Folha} & \textbf{Linha}  & \textbf{Onde se lê}  & \textbf{Leia-se}  \\
%\hline
%1 & 10 & auto-conclavo & autoconclavo\\
%\hline
%\end{tabular}
%\end{table}
%\end{errata}


% ----------------------------------------------------------
% Inserir folha de aprovação
% ----------------------------------------------------------

% Isto é um exemplo de Folha de aprovação, elemento obrigatório da NBR
% 14724/2011 (seção 4.2.1.3). Você pode utilizar este modelo até a aprovação
% do trabalho. Após isso, substitua todo o conteúdo deste arquivo por uma
% imagem da página assinada pela banca com o comando abaixo:
%
% \includepdf{folhadeaprovacao_final.pdf}
%
% Sugestão: Imprima uma cópia desta página e não encaderne-a, assim fica 
% mais fácil escaneá-la para gerar um pdf

\imprimefolhadeaprovacao


% ----------------------------------------------------------
% Dedicatória
% ----------------------------------------------------------
\begin{dedicatoria}
\vspace*{\fill}
\hspace{.24\textwidth}
\begin{minipage}{.7\textwidth}
\flushright
\noindent
\textit{Àqueles que um dia viajaram e deslumbraram \\
a beleza matemática.}
\end{minipage}
\end{dedicatoria}
%\imprimirdedicatoria

% ----------------------------------------------------------
% Agradecimentos
% ----------------------------------------------------------
\begin{agradecimentos}

\end{agradecimentos}


% ----------------------------------------------------------
% Epígrafe
% ----------------------------------------------------------
\begin{epigrafe}
\vspace*{\fill}
\hspace{.24\textwidth}
\begin{minipage}{.7\textwidth}
\flushright
\noindent
\textit{``Digno és, Jeová, nosso Deus, de receber a glória, a honra e o poder, porque criaste todas as coisas, e por tua vontade elas vieram à existência e foram criadas.''\\
(Apocalipse 4: 11)}
\end{minipage}
\end{epigrafe}


% ----------------------------------------------------------
% RESUMOS
% ----------------------------------------------------------

% resumo em português
\begin{resumo}


\vspace{\onelineskip}

\noindent
\textbf{Palavras-chaves}: Álgebras de Clifford. Álgebras Geométricas. Produto Exterior.
\end{resumo}

% resumo em inglês
\begin{resumo}[Abstract]
\begin{otherlanguage*}{english}
This is the english abstract.

\vspace{\onelineskip}
 
\noindent 
\textbf{Key-words}: Clifford Algebras. Geometric Algebras. Exterior Product.
\end{otherlanguage*}
\end{resumo}
% ---

% ---
% inserir lista de ilustrações
% ---
\pdfbookmark[0]{\listfigurename}{lof}
\listoffigures*
\newpage
% ---

% ---
% inserir lista de tabelas
% ---
\pdfbookmark[0]{\listtablename}{lot}
\listoftables*
\newpage
% ---

% ---
% inserir lista de abreviaturas e siglas
% ---
\begin{siglas}
\item[i. e.] \textit{id est} (isto é)
\end{siglas}
% ---

% ---
% inserir lista de símbolos
% ---
\begin{simbolos}
\item[$\Cldois$] Álgebra de Clifford
\item[$\Cldoispar$] Parte par de $\Cldois$
\item[$\Cldoisimpar$] Parte ímpar de $\Cldois$
\item[$\Cl_n$] 
\end{simbolos}
% ---

% ---
% inserir o sumario
% ---
\pdfbookmark[0]{\contentsname}{toc}
\tableofcontents*
\cleardoublepage
% ---


% ----------------------------------------------------------
% ELEMENTOS TEXTUAIS
% ----------------------------------------------------------
\textual
\pagestyle{simple}

\chapter{Introdução}

% ---
% Finaliza a parte no bookmark do PDF, para que se inicie o bookmark na raiz
% ---
\bookmarksetup{startatroot}% 
% ---

% ----------------------------------------------------------
% Conclusão
% ----------------------------------------------------------
\chapter{Conclusão}


% ----------------------------------------------------------
% ELEMENTOS PÓS-TEXTUAIS
% ----------------------------------------------------------
\postextual


% ----------------------------------------------------------
% Referências bibliográficas
% ----------------------------------------------------------
\bibliography{ref-algebrasdeclifford}


% ----------------------------------------------------------
% Apêndices
% ----------------------------------------------------------
%\begin{apendicesenv}
%\chapter{Quisque libero justo}
%\lipsum[50]
%
%\chapter{Nullam elementum urna vel imperdiet sodales}
%\lipsum[55-57]
%\end{apendicesenv}


% ----------------------------------------------------------
% Anexos
% ----------------------------------------------------------
%\begin{anexosenv}
%
%\chapter{Morbi ultrices rutrum lorem.}
%\lipsum[30]
%
%\chapter{Fusce facilisis lacinia dui}
%\lipsum[32]
%
%\end{anexosenv}

% ----------------------------------------------------------
% FIM DO DOCUMENTO
% ----------------------------------------------------------
\end{document}